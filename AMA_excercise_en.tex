\documentclass[a4paper,10pt]{book}
%\usepackage[active]{srcltx}
\usepackage[czech]{babel}
\usepackage[utf8]{inputenc}

\usepackage{amsmath}
\usepackage{amsfonts}
\usepackage{amssymb}
\usepackage{amsthm}
%
\newtheorem{theorem}{Věta}[section]
\newtheorem{proposition}[theorem]{Tvrzení}
\newtheorem{definition}[theorem]{Definice}
\newtheorem{remark}[theorem]{Poznámka}
\newtheorem{lemma}[theorem]{Lemma}
\newtheorem{corollary}[theorem]{Důsledek}
\newtheorem{exercise}[theorem]{Cvičení}

%\numberwithin{equation}{document}
%
\def\div{{\rm div}}
\def\Lapl{\Delta}
\def\grad{\nabla}
\def\supp{{\rm supp}}
\def\dist{{\rm dist}}
%\def\chset{\mathbbm{1}}
\def\chset{1}
%
\def\Tr{{\rm Tr}}
\def\to{\rightarrow}
\def\weakto{\rightharpoonup}
\def\imbed{\hookrightarrow}
\def\cimbed{\subset\subset}
\def\range{{\mathcal R}}
\def\leprox{\lesssim}
\def\argdot{{\hspace{0.18em}\cdot\hspace{0.18em}}}
\def\Distr{{\mathcal D}}
\def\calK{{\mathcal K}}
\def\FromTo{|\rightarrow}
\def\convol{\star}
\def\impl{\Rightarrow}
\DeclareMathOperator*{\esslim}{esslim}
\DeclareMathOperator*{\esssup}{ess\,supp}
\DeclareMathOperator{\ess}{ess}
\DeclareMathOperator{\osc}{osc}
\DeclareMathOperator{\curl}{curl}
\DeclareMathOperator{\cotg}{cotg}

%
%\def\Ess{{\rm ess}}
%\def\Exp{{\rm exp}}
%\def\Implies{\Longrightarrow}
%\def\Equiv{\Longleftrightarrow}
% ****************************************** GENERAL MATH NOTATION
\def\Real{{\rm\bf R}}
\def\C{{\rm\bf C}}
\def\Rd{{{\rm\bf R}^{\rm 3}}}
\def\RN{{{\rm\bf R}^N}}
\def\D{{\mathbb D}}
\def\Nnum{{\rm\bf N}}
\def\Qnum{{\rm\bf Q}}
\def\Measures{{\mathcal M}}
\def\d{\,{\rm d}}               % differential
\def\sdodt{\genfrac{}{}{}{1}{\rm d}{{\rm d}t}}
\def\dodt{\genfrac{}{}{}{}{\rm d}{{\rm d}t}}
%
\def\vc#1{\mathbf{\boldsymbol{#1}}}     % vector
\def\tn#1{{\mathbb{#1}}}    % tensor
\def\abs#1{\lvert#1\rvert}
\def\Abs#1{\bigl\lvert#1\bigr\rvert}
\def\bigabs#1{\bigl\lvert#1\bigr\rvert}
\def\Bigabs#1{\Big\lvert#1\Big\rvert}
\def\ABS#1{\left\lvert#1\right\rvert}
\def\norm#1{\bigl\Vert#1\bigr\Vert} %norm
\def\metr#1#2{\d\bigl(#1,#2\bigr)}          %metric
\def\close#1{\overline{#1}}
\def\inter#1{#1^\circ}
\def\eqdef{\mathrel{\mathop:}=}     % defining equivalence
\def\where{\,|\,}                    % "where" separator in set's defs
\def\timeD#1{\dot{\overline{{#1}}}}
%
% ******************************************* USEFULL MACROS
\def\RomanEnum{\renewcommand{\labelenumi}{\rm (\roman{enumi})}}   % enumerate by roman numbers
\def\rf#1{(\ref{#1})}                                             % ref. shortcut
\def\prtl{\partial}                                        % partial deriv.
\def\Names#1{{\scshape #1}}
\def\rem#1{{\parskip=0cm\par!! {\sl\small #1} !!}}
\def\vysl#1{\par$[$ #1 $]$}


\usepackage{paralist}
\usepackage{tabto}
\usepackage[inline]{enumitem}
\newenvironment{tabbedenum}[1]
 {\NumTabs{#1}\inparaenum\let\latexitem\item
  \def\item{\def\item{\tab\latexitem}\latexitem}}
{\endinparaenum}

 
%
%
% ******************************************* DOCUMENT NOTATIONS
% document specific
%***************************************************************************
%
\addtolength{\textwidth}{2cm}
\addtolength{\vsize}{2cm}
\addtolength{\topmargin}{-1cm}
\addtolength{\hoffset}{-1cm}
\begin{document}
\newcommand{\refbf}[1]{\textbf{\ref{#1}}}

\parskip=2ex

\parindent=0pt
\pagestyle{empty}


\chapter{Linear Algebra}

\section{Systems of linear equations}
\exercise \label{ex:slr1} Solve the given system of linear algebraic equations using the Gauss elimination.
\begin{center}
\begin{tabular}{ccc}
a)  $\begin{array}{rcl}
    2x+2y+5z & = & 11 \\
    x+y+z & = & 4 \\
    4x+6y+8z & = & 24 
  \end{array} $ &
b)  $\begin{array}{rcl}
    3x+y+z-2w & = & 6 \\
    2x-z+w & = & 9 \\
    x-2y+z+3w & = & 2 \\
    -3x+y+z-2w & = & -12
  \end{array} $ &
c)  $\begin{array}{rcl}
    2x - 3y - 2z + w & = & 3 \\
    x - y - z - w & = & 2 \\
    x - 2y - z + 2w & = & 1 \\
    2y + 2z + w & = & 1
  \end{array} $ \\  
d)  $\begin{array}{rcl}
    x - 2y + z - 3w & = & -3 \\
    x + y - 2z + 2w & = & 5 \\
    3x - 3z + w & = & 7 \\
    2x - y - z - w & = & 2
  \end{array} $ &
e) \begin{minipage}[h!]{5cm}
   $\, $\\
   $\mathbf{Ax} =\vc{b}$, where \\
   $\mathbf{A} = \begin{pmatrix} 
                      1 & 2 & 3 & 1 \\ 
                      2 & 4 & 7 & 7 \\
                      1 & 0 & 2 & 0 \\
                      3 & 7 & 10 & 6 \\
                  \end{pmatrix}$,\\
    $\vc{b}=(1,4,-2,7)^{T}$. 
    \end{minipage} &
\end{tabular}
\end{center}

\exercise \label{ex:slr2} Solve the given system of linear algebraic equations dependent on a parameter $\lambda\in\Real$.
\begin{center}
\begin{tabular}{ccc}
a)  $\begin{array}{rcl}
    \lambda x + y + z & = & 1 \\
    x + \lambda y + z & = & 1 \\
    x + y + \lambda z & = & 1 
  \end{array} $ %&
% b)  $\begin{array}{rcl}
%     3x+y+z-2w & = & 6 \\
%     2x-z+w & = & 9 \\
%     x-2y+z+3w & = & 2 \\
%     -3x+y+z-2w & = & -12
%   \end{array} $ &
% c)  $\begin{array}{rcl}
%     3x+y+z-2w & = & 6 \\
%     2x-z+w & = & 9 \\
%     x-2y+z+3w & = & 2 \\
%     -3x+y+z-2w & = & -12
%   \end{array} $   
\end{tabular}
\end{center}

\section{Matrices, eigenvalues and eigenvectors}
\exercise \label{ex:mvlc1} Compute the determinant of the given matrix $\mathbf{A}$ and the inverse $\mathbf{A}^{-1}$.

\begin{center}
\begin{tabular}{ccc}
a)  $ \mathbf{A}=
          \begin{pmatrix}
            1 & 6 & -16 \\
            -1 & -3 & 10 \\
            1 & 3 & -7
          \end{pmatrix}
  $ &
b)  $ \mathbf{A}=
          \begin{pmatrix}
            1 & 6 & -16 \\
            -1 & -3 & 10 \\
            1 & 3 & -7
          \end{pmatrix}
  $ &  
c)  $ \mathbf{A}=
          \begin{pmatrix}
            2 & 2 & 3 \\
            1 & -1 & 0 \\
            -1 & 2 & 1
          \end{pmatrix}
  $
\end{tabular}
\end{center}

\exercise \label{ex:mvlc2} Compute the eigenvalues and eigenvectors of the given matrix:

\begin{tabbedenum}{3}
\item $\begin{pmatrix} 1 & 2 \\ 4 & 3 \end{pmatrix}$
\item $\begin{pmatrix} 4 & 2 \\ -1 & 1 \end{pmatrix}$
\item $\begin{pmatrix} -3 & 4 & -2 \\ 1 & 0 & 1 \\ 6 & -6 & 5 \end{pmatrix}$
\item $\begin{pmatrix} -1 & 1 & 1 \\ 1 & -1 & 1 \\ 1 & 1 & 1 \end{pmatrix}$
\end{tabbedenum}

\section{Linear combination, vector space basis}

\exercise \label{ex:lkb1} Are the vectors linearly independent?

\begin{center}
\begin{tabular}{cccc}
a)  $ \begin{array}{rcl}
        \vec{x_1} &=& (1, -1, 2) \\
        \vec{x_2} &=& (-2, 3, 1) \\
        \vec{x_3} &=& (-1, 3, 8) 
      \end{array}
  $ &
b)  $ \begin{array}{rcl}
        \vec{x_1} &=& (1, -1, 2) \\
        \vec{x_2} &=& (-2, 3, 1) \\
        \vec{x_3} &=& (-1, 3, 7) \\
      \end{array}
  $ &  
c)  $ \begin{array}{rcl}
        \vec{x_1} &=& (1, 2, 4) \\
        \vec{x_2} &=& (2, 1, 3) \\
        \vec{x_3} &=& (4, -1, 1) \\
      \end{array}
  $ 
\end{tabular}
\end{center}

\exercise \label{ex:lkb2} Check if the vector $\vec{y}$ is a linear combination of the vector set $\vec{x_1}$, $\vec{x_2}$ and $\vec{x_3}$.

  \begin{enumerate*} 
    \item $\vec{y}=(2,8,12)$, $\vec{x_1}=(1,2,1)$, $\vec{x_2}=(3,2,-1)$ a $\vec{x_3}=(1,2,3)$  \\
    \item $\vec{y}=(2,8,8)$, $\vec{x_1}=(1,2,1)$, $\vec{x_2}=(3,3,5)$ a $\vec{x_3}=(1,-1,3)$  \\
  \end{enumerate*}
  
\exercise \label{ex:lkb3} Express the polynomial $p = x^2+x+1$ as a linear combination of the given polynomials
  \begin{eqnarray*}
    p_1 &=& 2x + 3 \\
    p_2 &=& x^2+2x+3 \\
    p_3 &=& -x^2+2x \\
    p_4 &=& -2x^2+x+2
  \end{eqnarray*}

\exercise \label{ex:lkb4} Can the given matrices make a basis of the space $\Real^{2\times2}$? 
If not, write an arbitrary basis of that space.
\vspace{-1em}
\begin{center}
$ \begin{pmatrix}
        1 & 1 \\ -3 & -4
  \end{pmatrix} \quad
  \begin{pmatrix}
        3 & 3 \\ 7 & 2
  \end{pmatrix} \quad
  \begin{pmatrix}
        2 & 1 \\ 1 & -1
  \end{pmatrix} \quad
  \begin{pmatrix}
        1 & 6 \\ 10 & 3
  \end{pmatrix}
$ 
\end{center}

\exercise \label{ex:lkb5} 
Let $V(\Omega)$ be a vector space of continuous functions defined in the given interval $\Omega$. 
Find out whether the functions $a,b,c\in V(\Omega)$ are linearly independent.

\begin{enumerate}[label=\alph*)]
\item $a = e^x$, $b = x$, $x \in \Omega=\Real$,
\item $a = x^2 +\frac{x}{2}-\frac{1}{2}$, $b=\frac{x^2}{3}+\frac{1}{3}$, 
      $c=x^2+x-2$, $x \in \Omega=\Real$,
\item $a = \sin x$, $b = \cos x$, $x \in \Omega=[0, 2\pi]$,
\item $a = 2\cos 2x$, $b = -\cos2x$, $c = -1$, $x \in \Omega=[-\pi, \pi]$.
\end{enumerate}

\section{Vector spaces}
\exercise \label{ex:vek1} 
Does the given set, closed under the defined addition and scalar multiplication, make a vector space?
\begin{equation}
  x\oplus y=x\cdot y, \quad \alpha \odot x = x^\alpha, \quad x,y \in \Real^{+}  \nonumber
\end{equation}

\exercise \label{ex:vek2} Let us define the set $V=\{(a,b,c)\where a+2b=0, c\in\Real\}\subset \Real^3$. 
Does the set $V$ make a vector space (a subspace of $\Real^3$)?

\exercise \label{ex:vek3} 
Let us define the set $V=\{(a,b,c)\where 2a-b-c=3\}\subset \Real^3$. 
Does the set $V$ make a vector space (a subspace of $\Real^3$)?

\exercise \label{ex:vek4} 
Check that the set of polynomials $M=\{p\in\mathcal{P}\where 2p(0)=p(1)\}$ is a subspace 
of the polynomial space $\mathcal{P}$.

\exercise \label{ex:vek5} 
Decide which of the following sets of functions, with the standard addition and scalar multiplication,
make a vector space.
\begin{enumerate}[label=\alph*)]
\item set of bounded functions on $[a, b]$,
\item set of increasing functions on $[a, b]$,
\item set of monotonic functions $[a, b]$,
\item set of even functions on $[-a, a]$, $a > 0$.
\end{enumerate}

\exercise \label{ex:vek6} 
Decide whether the given set is a vector space, considering the standard way of addition 
and scalar multiplication.
\begin{enumerate}[label=\alph*)]
\item the set of complex numbers $\rm\bf{C}$,
\item the set of real numbers $\Real$,
\item the set of positive real numbers $\Real^+$,
\item the set of rational numbers $\rm\bf{Q}$.
\end{enumerate}

\exercise \label{ex:vek7} 
Let $P$ be a set of all sequences of real numbers. Considering the standard way of addition
and scalar multiplication, is the set a vector space?
\begin{enumerate}[label=\alph*)]
\item $P$ is a set of sequences which converges to 0,
\item $P$ is a set of sequences which converges to 1,
\item $P$ is a set of all converging sequences.
\end{enumerate}

\exercise \label{ex:vek8} 
Check, whether the set of all polynomials $\mathcal{P}^{n}$ of the maximal order
$n$ makes a vector space?



%%%%%%%%%%%%%%%%%%%%%%%%%%%%%%%%%%%%%%%%%%%%%%%%%%%%%%%%%%%%%%%%%%%%%%%%%%%%%%%
%%%%%%                                  ŘEŠENÍ                          %%%%%%%
%%%%%%%%%%%%%%%%%%%%%%%%%%%%%%%%%%%%%%%%%%%%%%%%%%%%%%%%%%%%%%%%%%%%%%%%%%%%%%%
\section{Solution}

\refbf{ex:slr1} 
\begin{enumerate}[leftmargin=.75cm,align=left,label={\alph*)},itemsep=-5pt, topsep=-7pt]
    \item $S=\{(1,2,1)\}$
    \item $S=\{(3,1,-2,1)\}$  
    \item $S=\{\frac{1}{2}(t+5,6t+2,-7t-1,t)\where t\in\Real\}$    
    \item $S=\{(t,5t-4s-9,s,-3t+3s+7)\where t,s\in\Real\}$
    \item no solution, $S=\emptyset$
  \end{enumerate}
  
\refbf{ex:slr2} 
\begin{enumerate}[leftmargin=.75cm,labelsep=0cm,align=left,label={},itemsep=-5pt, topsep=-7pt]
 \item $\lambda_1=0,\, S_1=\{(1-r-s,r,s)\where r,s\in\Real\};\,$
 \item         $\lambda_2=-2,\, S_2=\emptyset; \,$
 \item    $\lambda_3\in\Real\setminus\{-2,1\},\, S_3=\{\frac{1}{\lambda+2}(1,1,1)\}$
\end{enumerate}

\refbf{ex:mvlc1} a) $\det\mathbf{A}=9$, $\mathbf{A}^{-1}=\frac{1}{3}\bigl(\begin{smallmatrix} -3 & -2 & 4\\ 1 & 3 & 2 \\ 0 & 1 & 1 \end{smallmatrix} \bigr)$  
               b) $\det\mathbf{A}=0$, $\mathbf{A}^{-1}$ does not exist
               c) $\det\mathbf{A}=-1$, $\mathbf{A}^{-1}=\bigl(\begin{smallmatrix} 1 & -4 & -3\\ 1 & -5 & -3 \\ -1 & 6 & 4 \end{smallmatrix} \bigr)$  
    
\refbf{ex:mvlc2} 
  \begin{enumerate}[leftmargin=.75cm,align=left,itemsep=-5pt, topsep=-7pt]
    \item $\lambda=\{5,-1\}$, $v=\{(1,2),(1,-1)\}$  
    \item $\lambda=\{2,3\}$, $v=\{(1,-1),(-2,1)\}$  
    \item $\lambda=\{1,-1,2\}$, $v=\{(1,1,0),(1,0,-1),(1,2,3)\}$    
    \item $\lambda=\{-1,2,-2\}$, $v=\{(-1,-1,1),(-1,1,0),(1,1,2)\}$
  \end{enumerate}

\refbf{ex:lkb1} a) no, b) yes, c) yes \\
\refbf{ex:lkb2} yes, with coefficients $a=(2,-1,3)$ \\
\refbf{ex:lkb3} $p = \alpha_i p_i$, where coefficients $\vec\alpha=\{(-\frac{1}{6}(17t+5),
              \frac{1}{6}(13t+7),\frac{1}{6}(t+1),t)\where t\in\Real\}$ \\
\refbf{ex:lkb4} no, they are linearly dependent; the standard base: 
$ \left\{\left(\begin{smallmatrix} 1 & 0 \\ 0 & 0
  \end{smallmatrix}\right),\,
  \left(\begin{smallmatrix}
        0 & 1 \\ 0 & 0
  \end{smallmatrix}\right),\,
  \left(\begin{smallmatrix}
        0 & 0 \\ 1 & 0
  \end{smallmatrix} \right),\,
  \left(\begin{smallmatrix}
        0 & 0 \\ 0 & 1
  \end{smallmatrix}\right)\right\}
$ \\
\refbf{ex:lkb5}
  \begin{enumerate}[label=\alph*), itemsep=-5pt, topsep=-7pt]
    \item independent
    \item dependent with coefficients $\vec\alpha=\{(-2t,3t,t)\where t\in\Real\}$
    \item independent
    \item dependent with coefficients $\vec\alpha=\{(t,-t,t)\where t\in\Real\}$
  \end{enumerate}

\refbf{ex:vek1} yes   \\
\refbf{ex:vek2} yes   \\
\refbf{ex:vek3} no, is not closed for $\oplus$ 
\refbf{ex:vek4} yes   \\
\refbf{ex:vek5} 
  \begin{enumerate}[label=\alph*), itemsep=-5pt, topsep=-7pt]
    \item yes
    \item no, for an increasing function $f(x)$, the function $kf(x)$ for $k<0$ is decreasing
    \item ne, např. pro funkce $f(x)=-x$ a $g(x)=x^2$ na intervalu $[0,1]$, je $f(x)+g(x)$   
          nemonotonní
    \item ano
  \end{enumerate}
\refbf{ex:vek6} 
  \begin{enumerate}[label=\alph*), itemsep=-5pt, topsep=-7pt]
    \item ano, $\vec{0}=0$
    \item ano, $\vec{0}=0$
    \item ne, pro $x\in\Real^{+}$, $k\in\Real$ neplatí $kx\in\Real^{+}$
    \item ne, pro $x\in\rm\bf{Q}$, $k\in\Real$ neplatí $kx\in\rm\bf{Q}$
  \end{enumerate}
\refbf{ex:vek7}
  \begin{enumerate}[label=\alph*), itemsep=-5pt, topsep=-7pt]
    \item ano    
    \item ne, $\lim\left((a)+(b)\right) = \lim (a) + \lim(b) = 2$
    \item ano
  \end{enumerate}



\chapter{Metrické, normované prostory, operátory, konvergence}
Norma splňuje:
\begin{enumerate}[label=N\arabic*, itemsep=-3pt, topsep=-7pt]
\item $\norm{x}=0$ \label{norm1}
\item $\norm{x}=\abs{\alpha}\norm{x}$ \label{norm2}
\item $\norm{x+y}\leq \norm{x} + \norm{y}$ \label{norm3}
\end{enumerate}

Metrika je generována normou $\metr{x}{y} = \norm{x-y}$ (neplatí opačně), splňuje:
\begin{enumerate}[label=M\arabic*, itemsep=-3pt, topsep=-7pt]
\item $\metr{x}{y} \geq 0$
\item $\metr{x}{y} = \metr{y}{x}$
\item $\metr{x}{y}\leq \metr{x}{z} + \metr{z}{y}$
\end{enumerate}

\vspace{0.5cm}

\exercise \label{ex:metr1} Ověřte, že množina $M \subset \Real^2, M=\{\vec{x}=(a,2a) \where a\in\Real\}$ s~normou 
  $\norm{(x_1,x_2)}=\abs{x_1}$ tvoří normovaný vektorový podprostor.
  
\exercise \label{ex:metr2} Ověřte, že množina $M \subset \Real^2$ s~normou 
  $\norm{(x_1,x_2)}=\left(\sqrt{\abs{x_1}} + \sqrt{\abs{x_2}}\right)^2$ tvoří normovaný vektorový podprostor.

\exercise \label{ex:norm1} Ověřte, že zadané funkce jsou normami nebo metrikami.

\begin{tabbedenum}{3}
\item $\norm{u}=\sqrt{\int \limits_a^b \abs{u(x)} \d x}$
\item $\norm{u}=\sqrt{\int \limits_a^b \abs{u(x)}^2 \d x}$
\item $\norm{u}=\sqrt{\int \limits_a^b \abs{u(x)}^3 \d x}$
\end{tabbedenum}

\exercise \label{ex:norm2} Ověřte, že množina funkcí $M \subset \C [0,1]$ s normou 
  $\norm{f}=\int \limits_0^1 \abs{f(x)}^2 \d x$ tvoří normovaný vektorový podprostor.

\exercise \label{ex:norm3} Ověřte, že množina polynomů $M = \{p(x)=ax^2+bx+c \where a,b,c \in \Real\}$ s~normou 
  $\norm{p}=\abs{p(0)}+\abs{p(1)}+\abs{p(2)}$ tvoří normovaný vektorový podprostor.

\exercise \label{ex:norm4} Ověřte, že množina polynomů $M = \{p(x)=ax^2+2(a+b)x+b \where a,b \in \Real\}$ s~normou 
  $\norm{p}=\abs{p'(-1)+p'(1)}+\abs{p''(0)}$ tvoří normovaný vektorový podprostor.
  
\exercise \label{ex:norm5} Ověřte, že množina polynomů $M = \{p(x)=ax^2+bx+b \where a,b \in \Real\}$ s~normou 
  $\norm{p}$ tvoří normovaný vektorový podprostor. 
\begin{enumerate}[label=\alph*), itemsep=-3pt, topsep=-7pt]
  \item $\norm{p}=\abs{p'(-1)+p'(1)}$
  \item $\norm{p}=\abs{p'(-1)+p'(1)}+\abs{p''(0)}$
  \item $\norm{p}=\abs{p'(-1)+p'(1)}+\abs{p''(0)}^{\frac{1}{2}}$
  \item $\norm{p}=\left(\sqrt{\abs{p'(-1)+p'(1)}}+\sqrt{\abs{p''(0)}}\right)^2$
\end{enumerate}

\exercise \label{ex:norm6} Ověřte, že množina polynomů $M = \{p(x)=ax^2+bx+c \where a,b,c \in \Real\}$ s~normou 
  $\norm{p}=\abs{\int \limits_0^1 p(x) \d x} + \sqrt{p(1)^2 + p(0)^2}$ tvoří normovaný vektorový podprostor.

\exercise \label{ex:norm7} Vypočítejte normu funkcí. 
\begin{enumerate}[label={\alph*)},itemsep=-2pt, topsep=-7pt]
         \item $L^1$, $L^\infty$, a $H^1$ normu pro funkci $f(x)=(x+1)(x-2)$ na intervalu $[-2,3]$.
         \item $\norm{f}_{L^1([0,2])}$ a $\norm{f}_{L^\infty([0,1])}$ pro $f(x)=-x(x-1)$
         \item $\norm{f}_{L^1([0,1])}$ a $\norm{f}_{L^2([0,1])}$ pro $f(x)=x^{-\frac12}$
         \item $\norm{f}_{L^2([0,2\pi])}$ pro $f(x) =\sin(kx)$ a $f(x)=\cos(kx)$ kde $k$ je libovolné celé číslo.
         \item $\norm{f}_{L^1([-\infty,\infty]}$ a $\norm{f}_{L^\infty([-\infty,\infty]}$  pro $f(x)=\frac{1}{1+x^2}$
         \item normu v prostoru $H^1([-1,1])$ pro funkce $\sqrt[3]{x}$ a $\sqrt[3]{x^2}$, pro které hodnoty parametru
                $p$ bude mít funkce $x^p$ konečnou normu?
\end{enumerate}
  
\exercise \label{ex:op1} Ověřte, zda zadaný operátor je symetrický a pozitivně definitní. \\
\begin{enumerate}[label=\alph*), itemsep=-3pt, topsep=-7pt]
\item $A(u)=-\frac{\partial u}{\partial x}$ na intervalu $\Omega=[0,1]$, $\Tr(\Omega)=0$.
\item $A(u)=-x\frac{\partial^2 u}{\partial x^2}-y \frac{\partial^2 u}{\partial y^2} + u \quad$ na $C^2_0(\Omega)$, $\Omega=(0,1)(0,1)$
\item $A(u)=\lambda(x,y)\frac{\partial^2 u}{\partial x^2} + \mu(x,y) \frac{\partial^2u}{\partial y^2} + u \quad$ na $C^2_0(\Omega)$, $\Omega=(0,1)(0,1)$
\item $A(u)=-4\frac{\partial^2 u}{\partial x^2}-\frac{\partial^2 u}{\partial y^2} + u \quad$ na $C^2_0(\Omega)$, $\Omega=(-2,2)(-2,2)$
\item $A(u)=-\frac{\partial}{\partial x}\left(2x\frac{\partial u}{\partial x}\right) 
            - \frac{\partial}{\partial x}\left(4y\frac{\partial u}{\partial y}\right) + u \quad$ na $C^2_0(\Omega)$, $\Omega=(0,2)(0,2)$
\end{enumerate}  

\exercise \label{ex:op2} Ověřte, zda zadaný operátor $A: X \to Y$ je lineární, symetrický a pozitivně definitní
na daném prostoru se skalárním součinem $Z$. \\
\begin{enumerate}[label={\alph*)},itemsep=-2pt, topsep=-7pt]
  \item \begin{equation}
          A:C^2((0,1))\to L^2((0,1)),\quad A(f)=\left(-\frac{\prtl^2 f}{\prtl x^2} +f \right),
          \quad Z=L^2((0,1)) \nonumber
        \end{equation}
  \item \begin{equation}
          A:C^2(\Omega)\to L^2(\Omega),\quad A(f)=\left(-\frac{\prtl^2 f}{\prtl x^2} -\frac{\prtl^2 f}{\prtl y^2} + f \right),
          \quad Z=L^2(\Omega),\quad \Omega= \text{  koule v }\Real^2 \nonumber
        \end{equation}
\end{enumerate} 
  
\exercise \label{ex:op3} Ověřte, zda zadaný operátor $A: X \to Y$ je lineární, symetrický a pozitivně definitní
na daném prostoru se skalárním součinem $Z$. Bude potřeba použít nerovnosti $(Af,f)_X \ge c(f',f')_X$ a $(f',f')_X \ge c(f,f)_X$.
\begin{enumerate}[label={\alph*)},itemsep=-2pt, topsep=-7pt]
  \item \begin{equation}
            A:C^2(\Omega)\to L^2(\Omega),\quad A(f)=\left(-2\frac{\prtl^2 f}{\prtl x^2} -3 \frac{\prtl^2 f}{\prtl y^2} \right)
            , \quad Z=L^2(\Omega),\quad \Omega=[0,1]\times[0,1] \nonumber
        \end{equation}
  \item \begin{equation}
            A:C^2((1,2))\to L^2((1,2)),\quad A(f)=-\frac{\prtl }{\prtl x}\left( x^2 \frac{\prtl f}{\prtl x} \right)+ f,
            \quad Z=L^2((1,2)) \nonumber
         \end{equation}
\end{enumerate} 

\exercise \label{ex:con1} Ověřte, zdali je zadaná posloupnost konvergentní v normě $L_1$ a $L_\infty$.
\begin{enumerate}[label=\alph*), itemsep=-3pt, topsep=-7pt]
  \item $f_n(x)=\cos\left(\frac{x}{n}\right)e^{-x}$ na $[-1,1]$
  \item $f_n(x)=2\cos\left(\frac{x}{n}\right)e^x$ na $[-1,1]$
  \item $f_n(x)=x^2+\frac{x}{n}$ na $[0,1]$
  \item $f_n(x)=x^n$ na $[0,1]$
  \item $f_n(x)=x^\frac{1}{2n-1}$ na $[-1,1]$
\end{enumerate}

\exercise \label{ex:sp1} Vyberte z uvedených normovaných prostorů neúplné a svou volbu podložte stručně okomentovaným příkladem. 
\[(Q,\norm{\cdot}),\; (\Real,\norm{\cdot}),\; (C,\norm{\cdot}_{L_1}),\;
      (C,\norm{\cdot}_{L_\infty}),\; (L_1,\norm{\cdot}_{L_1}),\; (L_2,\norm{\cdot}_{L_2}),\; (H^1,\norm{\cdot}_{H^1})\]

      
\exercise \label{ex:sp2} Definujte prostor $H^1(\Omega)$ a normu zavedenou pomocí standardního skalárního součinu.


%%%%%%%%%%%%%%%%%%%%%%%%%%%%%%%%%%%%%%%%%%%%%%%%%%%%%%%%%%%%%%%%%%%%%%%%%%%%%%%
%%%%%%                                  ŘEŠENÍ                          %%%%%%%
%%%%%%%%%%%%%%%%%%%%%%%%%%%%%%%%%%%%%%%%%%%%%%%%%%%%%%%%%%%%%%%%%%%%%%%%%%%%%%%
\section{Řešení}

\ref{ex:metr1} ano - uzavřenost pro sčítání, násobení reálnýn číslem, platí norma

\ref{ex:metr2} ne, neplatí \ref{norm3}

\ref{ex:norm1}
\begin{enumerate}[label={\alph*)},itemsep=-5pt, topsep=-7pt]
\item norma ne (\ref{norm2}), metrika ano
\item norma ano, metrika ano
\item norma ne, metrika ano
\end{enumerate}

\ref{ex:norm2} ne, neplatí \ref{norm3}

\ref{ex:norm3} ano

\ref{ex:norm4} ano

\ref{ex:norm6} ano

\ref{ex:norm7}
\begin{enumerate}[label=\alph*), itemsep=-3pt, topsep=-7pt]
  \item $1$, $\frac14$
  \item $2$, $\infty$
  \item $\sqrt{\pi}$
  \item $\pi$, $1$
  \item $\infty$, $\sqrt{(\frac67)^2+(6)^2}$, $p>\frac12$
\end{enumerate}

\ref{ex:op1}
\begin{enumerate}[label=\alph*), itemsep=-3pt, topsep=-7pt]
  \item ano
  \item není symetrický, ani poz. def.
  \item ano, pro $\lambda=konst.$ a $\lambda(y)>0$, $\mu(x) <0$
  \item ano
  \item není symetrický, ani poz. def.
\end{enumerate}

\ref{ex:op2} 
\begin{enumerate}[label=\alph*), itemsep=-5pt, topsep=-7pt]
  \item per partes
  \item Greenova věta
\end{enumerate}

\ref{ex:op3}
\begin{enumerate}[label=\alph*), itemsep=-5pt, topsep=-7pt]
  \item per partes v obou osách
  \item doplněná nerovnost (Poincareova)
\end{enumerate}

\ref{ex:con1}
\begin{enumerate}[label=\alph*), itemsep=-3pt, topsep=-7pt]
  \item $L_1$ ano, $L_\infty$ ano
  \item $L_1$ ano, $L_\infty$ ano
  \item $L_1$ ano, $L_\infty$ ano
  \item $L_1$ ano, $L_\infty$ ne
  \item $L_1$ ano, $L_\infty$ ne
\end{enumerate}


\chapter{Systems of ODEs, Principle fundamental system}

% ODR_soustavy_nehom.pdf
% is computed and checked
\exercise \label{ex:ode1} Compute the principle fundamental matrix of the ODE system, choose initial time 
$\tau=0$. (multiple eigenvalue)
\begin{eqnarray*}
\dot{x}_1 &=& -x_1  - x_2 + 5x_3 \\
\dot{x}_2 &=& -2x_1       + 6x_3 \\
\dot{x}_3 &=& -2x_1 - x_2 + 6x_3 
\end{eqnarray*}

% is computed and checked
\exercise \label{ex:ode11} Compute the principle fundamental matrix of the ODE system, choose initial time 
$\tau=0$. (multiple eigenvalue)
\begin{eqnarray*}
\vec{\dot{x}} = \begin{pmatrix} 0 & -1 & 1 \\ 2 & -3 & 1 \\ 1 & -1 & -1 \end{pmatrix} \vec{x}.
\end{eqnarray*}

% ODR_soustavy_nehom.pdf
\exercise \label{ex:ode2} Určete standardní fundamentální systém řešení soustavy
\begin{equation*}
\vec{\dot{x}} = \begin{pmatrix} 2 & 1 & 0 \\ 1 & 3 & -1 \\ -1 & 2 & 3 \end{pmatrix} \vec{x}.
\end{equation*}

% ODR_soustavy_nehom.pdf
\exercise \label{ex:ode3} Určete řešení soustavy pomocí standardního fundamentálního systému
\begin{eqnarray*}
\dot{x}_1 &=& x_2  + \cos t, \\
\dot{x}_2 &=& -x_1 + 1, 
\end{eqnarray*}
které splňuje počáteční podmínky $x_1(0)=1$ a $x_2(0)=1$.

% ODR_soustavy_nehom.pdf
\exercise \label{ex:ode4} Určete řešení soustavy pomocí standardního fundamentálního systému
\begin{eqnarray*}
\dot{x}_1 + 5x_1 + x_2 &=& e^t  ,\\
\dot{x}_2 + 3x_2 - x_1 &=& e^{2t} ,
\end{eqnarray*}
které splňuje počáteční podmínky $x_1(0)=1$ a $x_2(0)=1$.

% ODR_soustavy_nehom.pdf
\exercise \label{ex:ode5} Určete řešení soustavy pomocí standardního fundamentálního systému
\begin{eqnarray*}
\dot{x}_1 &=& x_2 - 2e^t ,\\
\dot{x}_2 &=& x_1 + t^2  ,
\end{eqnarray*}
které splňuje počáteční podmínky $x_1(0)=-2$ a $x_2(0)=1$.

% ODR_soustavy_nehom.pdf
\exercise \label{ex:ode6} Určete řešení soustavy pomocí standardního fundamentálního systému
\begin{eqnarray*}
\dot{x}_1 &=& 2x_1 - x_2 ,\\
\dot{x}_2 &=& -x_1 + 2x_2 - 5e^t \sin t ,
\end{eqnarray*}
které splňuje počáteční podmínky $x_1(0)=1$ a $x_2(0)=1$.

% Soustavy lineárních diferenciálních rovnic.pdf
% is computed and checked
\exercise \label{ex:ode7} Určete řešení soustavy pomocí standardního fundamentálního systému
\begin{eqnarray*}
\dot{x}_1 &=& -x_1 + x_2 -2e^{-t},\\
\dot{x}_2 &=& -6x_1 + 4x_2 - 4e^{-t},
\end{eqnarray*}
které splňuje počáteční podmínky 
\begin{eqnarray*}
  a)&\quad& x_1(0)=1,\; x_2(0)=1 \\
  b)&\quad& x_1(1)=1,\; x_2(1)=1 
\end{eqnarray*}

% is computed and checked
\exercise \label{ex:ode8} Solve the following system using the principle fundamental system
\begin{equation*}
\vec{\dot{x}} = \begin{pmatrix} 0 & 1 \\ -2 & -3 \end{pmatrix} \vec{x} + \begin{pmatrix} 1 \\ 2 \end{pmatrix} e^t, 
\qquad \vec{x}(0) = \begin{pmatrix} 1 \\ 1 \end{pmatrix}
\end{equation*}

% is computed and checked
\exercise \label{ex:ode9} Solve the following system using the principle fundamental system
\begin{equation*}
\vec{\dot{x}} = \begin{pmatrix} 0 & 3 \\ -1 & 4 \end{pmatrix} \vec{x} + \begin{pmatrix} 2 \\ -1 \end{pmatrix} e^{2t}, 
\qquad \vec{x}(0) = \begin{pmatrix} 1 \\ 1 \end{pmatrix}
\end{equation*}

% is computed and checked
\exercise \label{ex:ode10} Solve the following system using the principle fundamental system
\begin{equation*}
\vec{\dot{x}} = \begin{pmatrix} 0 & 1 \\ 12 & 1 \end{pmatrix} \vec{x} + \begin{pmatrix} -2 \\ 3 \end{pmatrix} e^t, 
\qquad \vec{x}(0) = \begin{pmatrix} 1 \\ -3 \end{pmatrix}
\end{equation*}

%%%%%%%%%%%%%%%%%%%%%%%%%%%%%%%%%%%%%%%%%%%%%%%%%%%%%%%%%%%%%%%%%%%%%%%%%%%%%%%
%%%%%%                                  ŘEŠENÍ                          %%%%%%%
%%%%%%%%%%%%%%%%%%%%%%%%%%%%%%%%%%%%%%%%%%%%%%%%%%%%%%%%%%%%%%%%%%%%%%%%%%%%%%%

\section{Solution}

\ref{ex:ode1} $\lambda=\{1,2,2\}$, 
              $\vec{v}_1=\left(\begin{smallmatrix} -1\\ 2 \\ 0 \end{smallmatrix} \right)$, 
              $\vec{v}_2=\left(\begin{smallmatrix} 1\\ 2 \\ 2 \end{smallmatrix} \right)$, 
              $\vec{v}_3=\left(\begin{smallmatrix} 1+t\\ 1+2t \\ 1+t \end{smallmatrix} \right)$,\\
              \hphantom{\ref{ex:ode1}} $\mathbf{U}(t,0)=\left(\begin{smallmatrix} 1 & 0 & -1\\ -2 & 0 & 2 \\ 0 & 0 & 0 \end{smallmatrix} \right)e^{t}
               +\left(\begin{smallmatrix} -2t & -t & 1+4t\\ 2-4t & 1-2t & -2+8t \\ -2t & -t & 1+4t \end{smallmatrix} \right)e^{2t}$
               
\ref{ex:ode11} $\lambda=\{-1,-1,-2\}$, 
              $\vec{v}_1=\left(\begin{smallmatrix} 0\\ 1 \\ 1 \end{smallmatrix} \right)$, 
              $\vec{v}_2=\left(\begin{smallmatrix} 1\\ 1 \\ 0 \end{smallmatrix} \right)$, 
              $\vec{v}_3=\left(\begin{smallmatrix} 1+t\\ 1+t \\ 1 \end{smallmatrix} \right)$,\\
              \hphantom{\ref{ex:ode1}} $\mathbf{U}(t,0)=\left(\begin{smallmatrix} 1 & 0 & 0\\ 1 & 0 & 0 \\ 1 & -1 & 1 \end{smallmatrix} \right)e^{-t}
               +\left(\begin{smallmatrix} 1 & -1 & 1\\ 1 & -1 & 1 \\ 0 & 0 & 0 \end{smallmatrix} \right)te^{-t}
               +\left(\begin{smallmatrix} 0 & 0 & 0\\ -1 & 1 & 0 \\ -1 & 1 & 0 \end{smallmatrix} \right)e^{-2t}$

\ref{ex:ode7} $\lambda=\{1,2\}$, 
              $\vec{v}_1=\left(\begin{smallmatrix} 1\\ 2 \end{smallmatrix} \right)$, 
              $\vec{v}_2=\left(\begin{smallmatrix} 1\\ 3 \end{smallmatrix} \right)$, \\
              \hphantom{\ref{ex:ode7}} $\mathbf{U}(t-\tau,0)=\left(\begin{smallmatrix} 3 & -1 \\ 6 & -2 \end{smallmatrix} \right)e^{t-\tau}
               +\left(\begin{smallmatrix} -2 & 1 \\ -6 & 3 \end{smallmatrix} \right)e^{2(t-\tau)}$
              
\begin{enumerate}[label=\alph*), itemsep=-3pt, topsep=-7pt]
  \item $\vec{x}=\left(\begin{smallmatrix} 1\\ 2 \end{smallmatrix} \right)\left(e^{t}+e^{-t}\right)
               -\left(\begin{smallmatrix} 1\\ 3 \end{smallmatrix} \right)e^{2t}$
  \item $\vec{x}=\left(\begin{smallmatrix} 1\\ 2 \end{smallmatrix} \right)\left(2e^{t-1}+e^{-t}-e^{t-2}\right)
               -\left(\begin{smallmatrix} 1\\ 3 \end{smallmatrix} \right)e^{2t-2}$
\end{enumerate}
               
\ref{ex:ode8} $\lambda=\{-1,-2\}$, 
              $\vec{v}_1=\left(\begin{smallmatrix} 1\\ -1 \end{smallmatrix} \right)$, 
              $\vec{v}_2=\left(\begin{smallmatrix} -1\\ 2 \end{smallmatrix} \right)$, 
              $\vec{x}=\left(\begin{smallmatrix} -1\\ 2 \end{smallmatrix} \right)e^{-2t}
               +\left(\begin{smallmatrix} 1\\ -1 \end{smallmatrix} \right)e^{-t}
               +\left(\begin{smallmatrix} 1\\ 0 \end{smallmatrix} \right)e^{t}$
               
\ref{ex:ode9} $\lambda=\{1,3\}$, 
              $\vec{v}_1=\left(\begin{smallmatrix} 3\\ 1 \end{smallmatrix} \right)$, 
              $\vec{v}_2=\left(\begin{smallmatrix} 1\\ 1 \end{smallmatrix} \right)$, 
              $\vec{x}=\frac{1}{2}\left[\left(\begin{smallmatrix} -3\\ -3 \end{smallmatrix} \right)e^{3t}
               +\left(\begin{smallmatrix} 14\\ 8 \end{smallmatrix} \right)e^{2t}
               -\left(\begin{smallmatrix} 9\\ 3 \end{smallmatrix} \right)e^{t}\right]$   
               
\ref{ex:ode10} $\lambda=\{-3,4\}$, 
              $\vec{v}_1=\left(\begin{smallmatrix} 1\\ -3 \end{smallmatrix} \right)$, 
              $\vec{v}_2=\left(\begin{smallmatrix} 1\\ 4 \end{smallmatrix} \right)$, 
              $\vec{x}=\frac{1}{28}\left[-4\left(\begin{smallmatrix} 1\\ 4 \end{smallmatrix} \right)e^{4t}
               +\left(\begin{smallmatrix} 39\\ -117 \end{smallmatrix} \right)e^{-3t}
               +\left(\begin{smallmatrix} -7\\ 49 \end{smallmatrix} \right)e^{t}\right]$   




\chapter{PDR}

\exercise \label{ex:pde1} Řešte metodou sítí rovnici 
\begin{equation*}
\frac{\partial^2 u}{\partial x^2} + \frac{\partial^2 u}{\partial y^2} = y-x \qquad \textrm{ na } \Omega \textrm{ s hranicí } \Gamma.
\end{equation*}
Oblast $\Omega$ je pětiúhelník s vrcholy $V_1[-1,0],\,V_2[-1,1.5],\,V_3[0,1.5],\,V_4[0.5,1]$ a $V_5[-0.5,0]$. Okrajová podmínka je zadána
\begin{equation*}
u(x,y) = y \qquad \textrm{ na } \Gamma.
\end{equation*}
Krok volte 0.5 v obou souřadnicích. Druhou derivaci aproximujte centrální diferencí.


\exercise \label{ex:pde2} Řešte metodou sítí rovnici 
\begin{equation*}
2\,\frac{\partial^2 u}{\partial x^2} + 3\,\frac{\partial^2 u}{\partial y^2} + u = (x-y)^2 \qquad \textrm{ na } \Omega=(0,1)\times(0,0.8).
\end{equation*}
Jsou zadány následující okrajové podmínky:
\begin{eqnarray*}
u(0,y) &=& -y \\
u(1,y) &=& 2y \\
u(x,0.8) &=& -0.8 + 2.4x \\
u(x,0) &=& 0
\end{eqnarray*}
Krok volte 0.25 v obou souřadnicích. Derivace aproximujte diferencí 2. řádu.

\exercise \label{ex:pde3} Řešte úlohu vedení tepla
\begin{equation*}
\frac{\partial u}{\partial t} = 0.3\,\frac{\partial^2 u}{\partial x^2} +x \qquad \textrm{ na } \Omega=(0,1)\times(0,0.4),
\end{equation*}
s počáteční podmínkou $u(x,0) = x^2$ pro $x\in [0,1]$ \\
a okrajovými podmínkami $u(0,t) = 0$, $u(1,t)=1$ pro $t>0$. \\
Zvolte prostorový krok 0.25 a časový krok co největší tak, aby numerické schéma bylo stabilní.

\exercise \label{ex:pde4} Řešte úlohu vedení tepla
\begin{equation*}
\frac{\partial u}{\partial t} = 0.2\,\frac{\partial^2 u}{\partial x^2} + 2t +x \qquad \textrm{ na } \Omega=(0,1)\times(0,T),
\end{equation*}
s počáteční podmínkou $u(x,0) = 0$ pro $x\in [0,1]$ \\
a okrajovými podmínkami $u(0,t) = 0$, $u(1,t)=3t$ pro $t>0$. \\
Zvolte prostorový krok 0.25 a časový krok 0.1. Ověřte, že explicitní numerické schéma bude stabilní a spočtěte aproximaci $u(0.75,0.4)$.






\end{document}
